%---------------------------Average Aspect Frobenius-----------------------------
\section{Mean Aspect Frobenius\label{s:hex-med-aspect-frobenius}}

For hexahedra, there is not a unique definition of the aspect Frobenius.
Instead, we use the aspect Frobenius
defined for tetrahedra (see section~\S\ref{s:tet-aspect-Frobenius}),
but choose the reference $W$ element to be right isosceles at
the hexahedral corner. Consider the eight tetrahedra formed by edges
incident to the corner of a hexahedron. 
Given a corner vertex $i$ and its three adjacent vertices $j$, $k$, and $\ell$ ordered
in a clockwise manner (so that $ijk\ell$ is a positively oriented tetrahedron),
denote the tetrahedral aspect frobenius of that corner as $F_{ijk\ell}$.
To obtain a single value for the metric, we average the eight unique tetrahedral aspects
\[
  q = \frac{1}{8}\left(F_{0134} + F_{1205} + F_{2316} + F_{3027} + F_{4750} + F_{5461} + F_{6572} + F_{7643} \right).
\]

\hexmetrictable{mean aspect frobenius}%
{$1$}%                                        Dimension
{$[1,3]$}%                                    Acceptable range
{$[1,DBL\_MAX]$}%                             Normal range
{$[1,DBL\_MAX]$}%                             Full range
{1}%                                          Cube
{--}%                                         Citation
{v\_hex\_med\_aspect\_frobenius}%             Verdict function name
